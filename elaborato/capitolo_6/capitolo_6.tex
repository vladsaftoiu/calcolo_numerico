\section{Capitolo 6}



	\subsection{Esercizio 6.1}
Scrivere una function Matlab che generi la matrice sparsa $n \times n$ con $n > 10$, 
$
	A = 
	\begin{pmatrix}
		a_{11} & ... & a_{1n} \\
		. &  & . \\
		a_{n1} & ... & a_{nn}
	\end{pmatrix}
	\quad \text{con} 
$, con $a_{ij} =$ 
\begin{empheq}[left=\empheqlbrace]{align*}
	4,  & \text{se} & i = j \\
	-1, & \text{se} & i = j  \\
	-1, & \text{se} & i = j 
\end{empheq}
Utilizzare, a questo fine, la function Matlab \lstinline{spdiags}.

La function che genera la matrice $A$ è la seguente:
\lstinputlisting{./capitolo_6/sparse_diags.m}



	\subsection{Esercizio 6.2}
Utilizzare il metodo delle potenze per calcolarne l’autovalore dominante della matrice $A_n$ del precedente esercizio, con una approssimazione $tol = 10^{-5}$, partendo da un vettore con elementi costanti. Riempire, quindi, la seguente
tabella:
\begin{tabular}{|c|c|c|}
\hline
n & numero di iterazioni effettuate & stima autovalore\\
\hline
100 & & \\
200 & & \\
... & & \\
1000 & & \\
\hline
\end{tabular}

\par
Il metodo delle potenze è implementato dalla seguente function:
\lstinputlisting{./capitolo_6/power_method.m}
la tabella viene generata usando lo script:
\lstinputlisting{./capitolo_6/exercise_6_2.m}
i risultati sono i seguenti:
\begin{tabular}{|c|c|r|}
	\hline
	n & iterazioni & stima $\lambda_1 $\\
	\hline
     	100  &  31  &  -5.9182  \\
     	200  &  49  &  -5.9760  \\
     	300  &  50  &  -5.9865  \\
     	400  &  44  &  -5.9905  \\
     	500  &  41  &  -5.9921  \\
     	600  &  40  &  -5.9928  \\
     	700  &  36  &  -5.9946  \\
     	800  &  37  &  -5.9944  \\
     	900  &  36  &  -5.9949  \\
    	1000  &  33  &  -5.9954  \\
    	\hline
\end{tabular}


	\subsection{Esercizio 6.3}

Utilizzare il metodo di Jacobi per risolvere il sistema lineare
\begin{equation*}
	A_n \mathbf{x} = \begin{pmatrix} 1 \\ \vdots \\ 1 \end{pmatrix}
\end{equation*}
dove $A_n$ è la matrice definita all'esercizio 6.1, con tolleranza $tol = 10^{-5}$, e partendo dal vettore nullo. Graficare il numero di iterazioni necessarie, rispetto alla dimensione $n$ del problema, con $n$ che varia da $100$ a $1000$ (con passo $20$).

La function che implementa il metodo di Jacobi è la seguente:
\lstinputlisting{./capitolo_6/jacobi.m}
lo script che la utilizza sugli $n = 100, 120, 140 ... 1000$ è:
\lstinputlisting{./capitolo_6/exercise_6_3.m}
che porta il seguente risultato:
\TODO[inserire grafico]


	\subsection{Esercizio 6.4}
Ripetere una procedura analoga a quella del precedente esercizio utilizzando il medodo di Gauss-Seidel.

La function che implementa il metodo di Gauss-Seidel è la seguente:
\lstinputlisting{./capitolo_6/gauss_seidel.m}
lo script che la utilizza sugli $n = 100, 120, 140 ... 1000$ è:
\lstinputlisting{./capitolo_6/exercise_6_4.m}
che porta il seguente risultato:
\TODO[inserire grafico]



	\subsection{Esercizio 6.5}
Con riferimento al sistema lineare
\begin{equation*}
	A_n \mathbf{x} = \begin{pmatrix} 1 \\ \vdots \\ 1 \end{pmatrix}
\end{equation*}
con $n = 1000$, graficare la norma dei residui, rispetto all’indice di iterazione, gererati dai metodi di Jacobi e GaussSeidel.
Utilizzare il formato \lstinline{semilogy} per realizzare il grafico, corredandolo di opportune \textit{labels}.

Lo script che utilizza le functions di Jacobi e Gauss-Seidel e ne stampa su un grafico la norma dei residui ad ogni iterazione è il seguente:
\lstinputlisting{./capitolo_6/exercise_6_5.m}