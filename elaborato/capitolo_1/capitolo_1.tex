\section{Capitolo 1}


	\subsection{Esercizio 1.1}

Sia $x = e \approx 2.7183 = \tilde{x}$. Si calcoli il corrispondente errore relativo $\varepsilon_x$ e il numero di cifre significative $k$ con cui $\tilde{x}$ approssima $x$. Si verifichi che:

\fek{\abs{\varepsilon_x} \approx \frac{1}{2}10^{-k}}

L'errore relativo è la quantità: $\varepsilon_x \equiv \frac{\Delta{x}}{x}=\frac{\tilde{x}-x}{x} = \frac{2.7183-e}{e} = 6.68494 \times 10^{-6}$.\\
Il numero di cifre significative $k$ è all'incirca $-\log_{10}{\abs{\varepsilon_x}}$ ovvero: 
\fek{-\log_{10}{6.68494 \times 10^{-6}}=5.174 \approx 5}.\\
Infatti le prime 5 cifre decimali dell'approssimazione $\tilde{x}$ sono corrette: $\tilde{x}  =\underline{2.7183}$ e $x = \underline{2.71828}182845...$.\\
Inoltre si verifica che: $\abs{\varepsilon_x} = \abs{6.685 \times 10^{-6}} = 0.6685 \times 10^{-5} \approx \frac{1}{2}10^{-5}$


	\subsection{Esercizio 1.2}
	
	Usando gli sviluppi di Taylor fino al secondo ordine con resto in forma di Lagrange, si verifiche che se $f \in C^3$, risulta:
	
\fek{f^{(1)}(x) = \phi_h(x) + O(h^2)}

	dove

\fek{\phi_h(x) = \frac{f(x+h)-f(x-h)}{2h}}

\TODO


	\subsection{Esercizio 1.3}
	
Utilizzando Matlab, si costruisca una tabella dove, per $h = 10^{-j}$, $j =1, . . . , 10$ e per la funzione $f(x) = x^{4}$ si riporta il valore di $\phi_h(x)$ definito nell'Esercizio 1 in $x = 1$. Commentare i risultati ottenuti.

\TODO


	\subsection{Esercizio 1.4}
	
Si dia una maggiorazione del valore assoluto dell’errore relativo con cui $x + y + z$ viene approssimato dall’approssimazione prodotta dal calcolatore, ossia $(x \oplus y) \oplus z$ (supporre che non ci siano problemi di overflow o di underflow). Ricavare l’analoga maggiorazione anche per $x \oplus (y \oplus z)$ tenendo presente che $x \oplus (y \oplus z) = (y \oplus z) \oplus x$.

\TODO


	\subsection{Esercizio 1.5}
Eseguire le seguenti istruzioni in Matlab:
\begin{lstlisting}[frame=single]
	x = 0; count = 0;	
	while x \tilde= 1, x = x + delta, count = count + 1, end
\end{lstlisting}
dapprima ponendo $delta = 1/16$ e poi ponendo $delta = 1/20$. Commentare i risultati ottenuti e in particolare il non funzionamento nel secondo caso.
\par
Il numero $1/16 = 0.0625$ in binario è: $0.0001$\\
Invece il numero $1/20 = 0.05$ in binario risulta periodico: $0.00\underline{0011}$, questo significa che, memorizzandolo in un area di memoria finita, si ha necessariamente perdita di informazione; in particolare:\\
IEEE 754 (base 2): $00111101010011001100110011001101$\\
ovvero in base 10: $0.0500000007450580596923828125 > 0.05$\\
pertanto $\phi_{20} = 1.0000000149 \neq 1$ e quindi il ciclo non concluderà mai.

	\subsection{Esercizio 1.6}
Verificare che entrambe le seguenti successioni convergono a $\sqrt{3}$ , (riportare le successive approssimazioni in una tabella a due colonne, una per ciascuna successione),
\fek{x_{k+1} = \frac{x_k + \frac{3}{x_k}}{2}, \quad x_0 = 3}
\fek{x_{k+1} = \frac{3+x_{k-1}x_k}{x_{k-1}+x_k}, \quad x_0 = 3, x_1=2}
Per ciascuna delle due successioni, dire quindi dopo quante iterazioni si ottiene un’approssimazione con un errore assoluto minore o uguale a $10^{-12}$ in valore assoluto.

\TODO