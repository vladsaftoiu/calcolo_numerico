\section{Capitolo 1}

	\subsection{Esercizio 1.1}

Sia $x = e \approx 2.7183 = \tilde{x}$. Si calcoli il corrispondente errore relativo $\varepsilon_x$ e il numero di cifre significative $k$ con cui $\tilde{x}$ approssima $x$. Si verifichi che:

\begin{equation}
	\abs{\varepsilon_x} \approx \frac{1}{2}10^{-k}
\end{equation}

L'errore relativo è la quantità: $\varepsilon_x \equiv \frac{\Delta{x}}{x}=\frac{\tilde{x}-x}{x} = \frac{2.7183-e}{e} = 6.68494 \times 10^{-6}$.\\
Il numero di cifre significative $k$ è all'incirca $-\log_{10}{\abs{\varepsilon_x}}$ ovvero: $-\log_{10}{6.68494 \times 10^{-6}}=5.174 \approx 5$.\\
Infatti le prime 5 cifre decimali dell'approssimazione $\tilde{x}$ sono corrette: $\tilde{x}  =\underline{2.7183}$ e $x = \underline{2.71828}182845...$.\\
Inoltre si verifica che: $\abs{\varepsilon_x} = \abs{6.685 \times 10^{-6}} = 0.6685 \times 10^{-5} \approx \frac{1}{2}10^{-5}$

	\subsection{Esercizio 1.2}

Utilizzare l'algoritmo 3.6 del libro per stabilire se le seguenti matrici sono $sdp$ o no:
	\[	
		A_1 = 
		\begin{pmatrix}
			1&-1&2&2 \\
			-1&5&-14&2\\
			2&-14&42&2\\
			2&2&2&65\\
		\end{pmatrix}
	\]
	\[		
		A_2 =
		\begin{pmatrix}
			1&-1&2&2\\
			-1&6&-17&3\\
			2&-17&48&-16\\
			2&3&-16&4\\
		\end{pmatrix}
	\]

Usando lo script: \\
\lstinputlisting{./capitolo_1/sdp_test.m}
che si appoggia all'algoritmo 3.6:\\
\lstinputlisting{./capitolo_3/LDL.m}
I risultati sono i seguenti:\\
\begin{lstlisting}[frame=single]
sdp_test

A =

     1    -1     2     2
    -1     5   -14     2
     2   -14    42     2
     2     2     2    65

la matrice e' SDP

A =

     1    -1     2     2
    -1     6   -17     3
     2   -17    48   -16
     2     3   -16     4

la matrice NON e' SDP
\end{lstlisting}