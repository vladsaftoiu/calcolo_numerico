\section{Capitolo 5}



	\subsection{Esercizio 5.1}
Scrivere una function Matlab che implementi la formula composita dei trapezi su $n+1$ ascisse equidistanti nell'intervallo $[a, b]$, relativamente alla funzione implementata da \texttt{fun(x)}. 
La function deve essere del tipo: \texttt{ [If] = trapcomp( n, a, b, fun ) }.
La formula composita dei trapezi è così definita:
\begin{equation}\label{trapezi_composite_equation}
	I_1^{(n)} = \frac{b-a}{2n} (f_0 + 2\sum_{i=1}^{n-1}f_i + f_n)
\end{equation}
La function Matlab che implementa la \ref{trapezi_composite_equation} è:
\lstinputlisting{./capitolo_5/trapcomp.m}



	\subsection{Esercizio 5.2}
Scrivere una function Matlab che implementi la formula composita di Simpson sulle $2n+1$ ascisse equidistanti nell'intervallo $[a, b]$, relativamente alla funzione implementata da \texttt{fun(x)}. 
La function deve essere del tipo: \texttt{ [If] = simpcomp( n, a, b, fun ) }.
La formula composita di Simpson è così definita:
\begin{equation}\label{simpson_composite_equation}
	I_2^{(n)} = \frac{b-a}{3n} (4\sum_{i=1}^{\frac{n}{2}}f_{2i-1} +2\sum_{i=0}^{\frac{n}{2}}f_{2i} - f_0 - f_n)
\end{equation}
La function Matlab che implementa la \ref{simpson_composite_equation} è:
\lstinputlisting{./capitolo_5/simpcomp.m}



	\subsection{Esercizio 5.3}
	Scrivere una function Matlab che implementi la formula composita dei trapezi adattativa nell’intervallo $\lbrack a,b \rbrack$  relativamente alla funzione implementata da \lstinline{fun(x)}, e con tolleranza \lstinline{tol}. La function deve essere del tipo: \lstinline{If = trapad( a, b, fun, tol )}
	
	La function è la seguente:
\lstinputlisting{./capitolo_5/trapad.m}
La function prevede anche la possibilità di calcolare il numero di valutazioni di funzione necesssarie per raggiungere l'integrale della funzione entro una certa tolleranza.



	\subsection{Esercizio 5.4}
	Scrivere una function Matlab che implementi la formula composita di Simpson adattativa nell’intervallo $\lbrack a,b \rbrack$  relativamente alla funzione implementata da \lstinline{fun(x)}, e con tolleranza \lstinline{tol}. La function deve essere del tipo: \lstinline{If = simpad( a, b, fun, tol )}
	
		La function è la seguente:
\lstinputlisting{./capitolo_5/simpad.m}
La function prevede anche la possibilità di calcolare il numero di valutazioni di funzione necesssarie per raggiungere l'integrale della funzione entro una certa tolleranza.



	\subsection{Esercizio 5.5}
	Calcolare quante valutazioni di funzione sono necessarie per ottenere una approssimazione di
\begin{equation*}
	\mathcal{I}(f) = \int_{0}^{1} e^{-10^6x}dx
\end{equation*}
che vale $10^{-6}$, in doppia precisione IEEE, con una tolleranza di $10^{-9}$, utilizzando le functions dei precedenti esercizi. Argomentare quantitativamente la risposta.

Utilizzando le functions \lstinline{ trapcomp, simpcomp,  trapad, simpad } precedentemente esposte si verifica, tramite lo script 
\lstinputlisting{./capitolo_5/integrals_test.m}
 che il numero di valutazioni di funzione sono: 
\begin{tabular}{c | r}
	\textbf{function} & \textbf{nr. evals}\\
	\hline
	trapcomp & $\approx 10^{6}$ \\
	simpcomp & $\approx 10^{5}$ \\
	trapad & 8527 \\ 
	simpad & 217 
\end{tabular}