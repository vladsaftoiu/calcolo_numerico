\documentclass{article}

\usepackage{bbm}
\usepackage{amsmath}
\usepackage{mathtools}
\usepackage{empheq}
\usepackage[italian]{babel}
\usepackage[utf8]{inputenc}
\usepackage{physics}
\usepackage{multirow}

\title{Calcolo Numerico - die Beziehung}
\date{10 giugno 2018}
\author{Saftoiu Vlad Alexandru}

\usepackage{tkz-graph}
\usepackage{listings}
\usepackage{color}
\usepackage{graphicx}
\usepackage{framed}

\definecolor{shadecolor}{rgb}{1,1,0.5}
\definecolor{dkgreen}{rgb}{0,0.6,0}
\definecolor{gray}{rgb}{0.5,0.5,0.5}
\definecolor{mauve}{rgb}{0.58,0,0.82}

\DeclarePairedDelimiter{\ceil}{\lceil}{\rceil}

\lstset{frame=tb,
  language=Matlab,
  aboveskip=3mm,
  belowskip=3mm,
  showstringspaces=false,
  columns=flexible,
  basicstyle={\small\ttfamily},
  numbers=none,
  numberstyle=\tiny\color{gray},
  keywordstyle=\color{blue},
  commentstyle=\color{dkgreen},
  stringstyle=\color{mauve},
  breaklines=true,
  breakatwhitespace=true,
  tabsize=3
}


\begin{document}
	
	\newcommand{\TODO}[0]{
		\begin{shaded*} 
			da fare
		\end{shaded*}
	}
	
	\newcommand{\fek}[1]{
		\begin{equation*}
			#1
		\end{equation*}
	}

	\pagenumbering{gobble}
	\maketitle
	\newpage
	\pagenumbering{arabic}

	\tableofcontents
	\newpage
	
	\section{Capitolo 1}


	\subsection{Esercizio 1.1}

Sia $x = e \approx 2.7183 = \tilde{x}$. Si calcoli il corrispondente errore relativo $\varepsilon_x$ e il numero di cifre significative $k$ con cui $\tilde{x}$ approssima $x$. Si verifichi che:

\begin{equation}
	\abs{\varepsilon_x} \approx \frac{1}{2}10^{-k}
\end{equation}

L'errore relativo è la quantità: $\varepsilon_x \equiv \frac{\Delta{x}}{x}=\frac{\tilde{x}-x}{x} = \frac{2.7183-e}{e} = 6.68494 \times 10^{-6}$.\\
Il numero di cifre significative $k$ è all'incirca $-\log_{10}{\abs{\varepsilon_x}}$ ovvero: $-\log_{10}{6.68494 \times 10^{-6}}=5.174 \approx 5$.\\
Infatti le prime 5 cifre decimali dell'approssimazione $\tilde{x}$ sono corrette: $\tilde{x}  =\underline{2.7183}$ e $x = \underline{2.71828}182845...$.\\
Inoltre si verifica che: $\abs{\varepsilon_x} = \abs{6.685 \times 10^{-6}} = 0.6685 \times 10^{-5} \approx \frac{1}{2}10^{-5}$


	\subsection{Esercizio 1.2}

Utilizzare l'algoritmo 3.6 del libro per stabilire se le seguenti matrici sono $sdp$ o no:
	\[	
		A_1 = 
		\begin{pmatrix}
			1&-1&2&2 \\
			-1&5&-14&2\\
			2&-14&42&2\\
			2&2&2&65\\
		\end{pmatrix}
	\]
	\[		
		A_2 =
		\begin{pmatrix}
			1&-1&2&2\\
			-1&6&-17&3\\
			2&-17&48&-16\\
			2&3&-16&4\\
		\end{pmatrix}
	\]

Usando lo script: \\
\lstinputlisting{./capitolo_1/sdp_test.m}
che si appoggia all'algoritmo 3.6:\\
\lstinputlisting{./capitolo_3/LDL.m}
I risultati sono i seguenti:\\
\begin{lstlisting}[frame=single]
sdp_test

A =

     1    -1     2     2
    -1     5   -14     2
     2   -14    42     2
     2     2     2    65

la matrice e' SDP

A =

     1    -1     2     2
    -1     6   -17     3
     2   -17    48   -16
     2     3   -16     4

la matrice NON e' SDP
\end{lstlisting}


	\subsection{Esercizio 1.3}

Scrivere una function Matlab che, avendo in ingresso un vettore $b$ contenente i termini noti del sistema lineare $Ax = b$ con $A$ sdp e l’output dell’Algoritmo 3.6 del libro (matrice $A$ riscritta nella porzione triangolare inferiore con i fattori $L$ e $D$ della fattorizzazione $LDL^{T}$ di $A$), ne calcoli efficientemente la soluzione.

\TODO
	

	\subsection{Esercizio 1.4}

Scrivere una function Matlab che, avendo in ingresso un vettore $b$ contenente i termini noti del sistema lineare $Ax = b$ e l’output dell’Algoritmo 3.7 del libro (matrice $A$ riscritta con la fattorizzazione $LU$ con pivoting parziale e il vettore $p$ delle permutazioni), ne calcoli efficientemente la soluzione.

\TODO


	\subsection{Esercizio 1.5}

 Inserire alcuni esempi di utilizzo delle due function implementate per i punti 3 e 4, scegliendo per ciascuno di essi un vettore $\tilde{x}$ e ponendo $b = A\tilde{x}$. Riportare $\tilde{x}$ e la soluzione $x$ da essi prodotta. Costruire anche una tabella in cui, per ogni esempio considerato, si riportano il numero di condizionamento di $A$ in norma 2 (usare $cond$ di Matlab) e le quantità $\frac{\norm{r}}{\norm{b}}$ e $\frac{\norm{x-\tilde{x}}}{\norm{\tilde{x}}}$.

\TODO
    \section{Capitolo 2}


	\subsection{Esercizio 2.1}

Determinare analiticamente gli zeri del polinomio 
\begin{equation*}
	P(x) = x^3 - 4x^2 + 5x - 2
\end{equation*}
 e la loro molteplicità. Dire perché il metodo di bisezione è utilizzabile per approssimarne uno a partire dall’intervallo di confidenza $[a,b]=[0,3]$. A quale zero di P potrà tendere la successione generata dal metodo di bisezione a partire da tale intervallo? Costruire una tabella in cui si riportano il numero di iterazioni e di valutazioni di P richieste per valori decrescenti della tolleranza $tolx$.

Dato che $P(x) = x^3 - 4x^2 + 5x - 2 = (x-2)(x-1)^2$ le radici di $P(x)$ sono $x_1=1$ e $x_2=2$. Rispetto all'intervallo $[a,b]=[0,3]$ si ha che $P(0) = -2$ e $P(3)=4$ ovvero il segno è discorde pertando, data la continuità della funzione, $P(x)$ assumerà sicuramente il valore $0$ in tale intervallo; è quindi possibile usare il metodo di bisezione partendo con $[a,b] = [0,3].$
Il metodo di bisezione ha il pregio di poter calcolare il numero massimo di iterazioni a priori; dati $a,b$ estremi dell'intervallo e $tolx$ valore della tolleranza richiesta, la formula corrispondente è la seguente:
\begin{equation*}
	imax= \ceil{\log_2{(b-a)}-\log_2{(tolx)}}
\end{equation*}
Considerato che il primo punto intermedio è $x=3/2$ si ha $P(3/2)P(3)<0$; il metodo convergerà quindi verso la radice situata nella seconda metà dell'intervallo, ovvero $x_2=2$.

Usando il seguente algoritmo per il metodo di bisezione:
\lstinputlisting{./capitolo_2/bisection.m}
si osserva che le valutazioni le valutazioni di funzione sono sempre $2+i$, ovvero le due iniziali più una ad ogni iterazione. La tabella sarà quindi:

\begin{tabular}{ c | c | c }
$tolx$ & imax & feval \\
\hline
$10^{-1}$ & 5 & 7 \\
$10^{-2}$ & 9 & 11 \\
$10^{-3}$ & 12 & 14 \\
$10^{-4}$ & 15 & 17 \\
$10^{-5}$ & 19 & 21 \\
$10^{-6}$ & 22 & 24 \\
$10^{-8}$ & 29 & 31 \\
$10^{-9}$ & 32 & 34 \\
$10^{-10}$ & 35 & 37 \\
$10^{-15}$ & 52 & 54 \\
$10^{-20}$ & 69 & 71 \\
$10^{-25}$ & 85 & 87 \\
$10^{-30}$ & 102 & 104 \\
$10^{-40}$ & 135 & 137 \\
$10^{-50}$ & 168 & 170 \\
$10^{-60}$ & 201 & 203 \\
$10^{-70}$ & 235 & 237 \\
$10^{-80}$ & 268 & 270 

\end{tabular}


	\subsection{Esercizio 2.2}
	
Completare la tabella precedente riportando anche il numero di iterazioni e di valutazioni di $P$ richieste dal metodo di Newton, dal metodo delle corde e dal metodo delle secanti (con secondo termine della successione ottenuto con Newton) a partire dal punto $x_0=3$. Commentare i risultati riportati in tabella; è possibile utilizzare $x_0 = \frac{5}{3}$ come punto di innesco?

\TODO



	\subsection{Esercizio 2.3}

Costruire una seconda tabella analoga alla precedente relativa ai metodi di Newton, di Newton modificato e di accelerazione di Aitken applicati alla funzione polinomiale P a partire dal punto di innesco $x_0 = 0$. Commentare i risultati riportati in tabella.

\TODO



	\subsection{Esercizio 2.4}

Definire una procedura iterativa basata sul metodo di Newton per approssimare $\sqrt{\alpha}$ per un assegnato $\alpha > 0$. Costruire una tabella dove
si riportano le successive approssimazioni ottenute e i corrispondenti errori assoluti (usare l’approssimazione Matlab di $\sqrt{\alpha}$ per il calcolo dell'errore) nel caso in cui $\alpha = 5$ partendo da $x_0 = 5$.

\TODO



	\subsection{Esercizio 2.4}

Definire una procedura iterativa basata sul metodo delle secanti sempre per approssimare $\sqrt{\alpha}$ per un assegnato $\alpha > 0$. Completare la tabella
precedente aggiungendovi i risultati ottenuti con tale procedura partendo da $x_0 = 5$ e $x_1=3$. Commentare i risultati riportati in tabella.


\fek{\abs{\varepsilon_x} \approx \frac{1}{2}10^{-k}}

L'errore relativo è la quantità: $\varepsilon_x \equiv \frac{\Delta{x}}{x}=\frac{\tilde{x}-x}{x} = \frac{2.7183-e}{e} = 6.68494 \times 10^{-6}$.\\
Il numero di cifre significative $k$ è all'incirca $-\log_{10}{\abs{\varepsilon_x}}$ ovvero: 
\fek{-\log_{10}{6.68494 \times 10^{-6}}=5.174 \approx 5}.\\
Infatti le prime 5 cifre decimali dell'approssimazione $\tilde{x}$ sono corrette: $\tilde{x}  =\underline{2.7183}$ e $x = \underline{2.71828}182845...$.\\
Inoltre si verifica che: $\abs{\varepsilon_x} = \abs{6.685 \times 10^{-6}} = 0.6685 \times 10^{-5} \approx \frac{1}{2}10^{-5}$


	\subsection{Esercizio 1.2}
	
	Usando gli sviluppi di Taylor fino al secondo ordine con resto in forma di Lagrange, si verifiche che se $f \in C^3$, risulta:
	
\fek{f^{(1)}(x) = \phi_h(x) + O(h^2)}

	dove

\fek{\phi_h(x) = \frac{f(x+h)-f(x-h)}{2h}}

\TODO


	\subsection{Esercizio 1.3}
	
Utilizzando Matlab, si costruisca una tabella dove, per $h = 10^{-j}$, $j =1, . . . , 10$ e per la funzione $f(x) = x^{4}$ si riporta il valore di $\phi_h(x)$ definito nell'Esercizio 1 in $x = 1$. Commentare i risultati ottenuti.

Utilizzando lo script:
\lstinputlisting{./capitolo_1/esercizio_3.m}
si ottengono i seguenti risultati:
\begin{tabular}{ c | c }
i & $\phi_h(1)$ \\
\hline
1  & 4.040000000000002e+00 \\
2  & 4.000400000000004e+00 \\
3  & 4.000003999999723e+00 \\
4  & 4.000000039999230e+00 \\
5  & 4.000000000403681e+00 \\
6  & 3.999999999948489e+00 \\
7  & 4.000000000115023e+00 \\
8  & 4.000000003445692e+00 \\
9  & 4.000000108916879e+00 \\
10 & 4.000000330961484e+00 \\
\end{tabular}

	\subsection{Esercizio 1.4}
	
Si dia una maggiorazione del valore assoluto dell’errore relativo con cui $x + y + z$ viene approssimato dall’approssimazione prodotta dal calcolatore, ossia $(x \oplus y) \oplus z$ (supporre che non ci siano problemi di overflow o di underflow). Ricavare l’analoga maggiorazione anche per $x \oplus (y \oplus z)$ tenendo presente che $x \oplus (y \oplus z) = (y \oplus z) \oplus x$.

Tenendo presente la funzione $fl: \mathrm{I} \rightarrow \mathrm{M}$ definita come: $fl(x) \equiv \tilde{x} = x(1+\varepsilon_x)$ e considerando che in aritmentica esatta $(x+y)+z = x+y+z$, l'errore relativo è misurato come $\frac{\tilde{x}-x}{x} = \frac{(x \oplus y) \oplus z - x - y - z}{x+y+z}$ e quindi si ha:
\begin{equation*}
	\begin{split}
		(x \oplus y) \oplus z \\
			& = fl \lbrack \; fl(fl(x) + fl(y)) + fl(z) \;\rbrack \\
			& = fl \lbrack \; fl(x(1+\varepsilon_x) + y(1+\varepsilon_y)) + z(1+\varepsilon_z) \; \rbrack \\
			& = fl \{ \; \lbrack \; x(1+\varepsilon_x) + y(1+\varepsilon_y) \; \rbrack (1 + \varepsilon_{\alpha}) + z(1+\varepsilon_z) \; \} \\
	  		& = \{ \; \lbrack \; x(1+\varepsilon_x) + y(1+\varepsilon_y) \; \rbrack (1 + \varepsilon_{\alpha}) + z(1+\varepsilon_z) \; \} (1 + \varepsilon_{\beta}) \\
	\end{split}
\end{equation*}

\begin{equation*}
	\begin{split}
		\frac{(x \oplus y) \oplus z - x - y - z}{x+y+z} \\
			& \leq \frac{x(1+u)^3 + y(1+u)^2 + z(1+u^2) -x -y -z}{x+y+z} \\
	 		& = \frac{x(3u+3u^2+u^3) + y(3u+3u^2+u^3) + z(2u+u^2)}{x+y+z} \\
	 		& \leq \frac{x(3u+3u^2+u^3) + y(3u+3u^2+u^3) + z(2u+u^2)}{x+y+z} \\
	 		& \leq \frac{7ux+7uy+3uz}{x+y+z} \\
	 		& = u(3+4 \frac{x+y}{x+y+z})
	\end{split}
\end{equation*}


	\subsection{Esercizio 1.5}
Eseguire le seguenti istruzioni in Matlab:
\begin{lstlisting}[frame=single]
	x = 0; count = 0;	
	while x \tilde= 1, x = x + delta, count = count + 1, end
\end{lstlisting}
dapprima ponendo $delta = 1/16$ e poi ponendo $delta = 1/20$. Commentare i risultati ottenuti e in particolare il non funzionamento nel secondo caso.
\par
Il numero $1/16 = 0.0625$ in binario è: $0.0001$\\
Invece il numero $1/20 = 0.05$ in binario risulta periodico: $0.00\underline{0011}$, questo significa che, memorizzandolo in un area di memoria finita, si ha necessariamente perdita di informazione; in particolare:\\
IEEE 754 (base 2): $00111101010011001100110011001101$\\
ovvero in base 10: $0.0500000007450580596923828125 > 0.05$\\
pertanto $\phi_{20} = 1.0000000149 \neq 1$ e quindi il ciclo non concluderà mai.

	\subsection{Esercizio 1.6}
Verificare che entrambe le seguenti successioni convergono a $\sqrt{3}$ , (riportare le successive approssimazioni in una tabella a due colonne, una per ciascuna successione),
\fek{x_{k+1} = \frac{x_k + \frac{3}{x_k}}{2}, \quad x_0 = 3}
\fek{x_{k+1} = \frac{3+x_{k-1}x_k}{x_{k-1}+x_k}, \quad x_0 = 3, x_1=2}
Per ciascuna delle due successioni, dire quindi dopo quante iterazioni si ottiene un’approssimazione con un errore assoluto minore o uguale a $10^{-12}$ in valore assoluto.

\par
Usando lo script:
\lstinputlisting{./capitolo_1/esercizio_6.m}
si ottengono i seguenti risultati:

\begin{tabular}{ r | c | c | c | c }
		
  \textbf{i} & \textbf{$s_1$} & \textbf{$e_1$} & \textbf{$s_2$} & \textbf{$e_2$} \\
  \hline	
  	1 & 3.0000000 &                & 3.0000000 &               \\
	2 & 2.0000000 & 2.6794919 e-01 & 2.0000000 &               \\
	3 & 1.7500000 & 1.7949192 e-02 & 1.8000000 & 6.7949192 e-02\\
	4 & 1.7321429 & 9.2049574 e-05 & 1.7368420 & 4.7912977 e-03\\
	5 & 1.7320508 & 2.4458502 e-09 & 1.7321429 & 9.2049574 e-05\\
	6 & 1.7320508 &                & 1.7320509 & 1.2713716 e-07\\
	7 &           &                & 1.7320508 & 3.3786307 e-12\\
	8 &           &                & 1.7320508 & 2.2204460 e-16\\
  \hline  
\end{tabular}

Si verifica che entrambe le successioni convergono al valore $\sqrt{3}$, la prima successione converge più velocemente e l'errore assoluto sarà minore o uguale a $10^{-12}$ dopo la 5a iterazione, mentre per la seconda successione questo avviene dopo la 7a iterazione.
	\section{Capitolo 3}



	\subsection{Esercizio 3.1}

Scrivere una function Matlab per la risoluzione di un sistema lineare con matrice dei coefficienti triangolare inferiore a diagonale unitaria. Inserire un esempio di utilizzo.
\PP
La function per risolvere il sistema su indicato è la seguente:

\lstinputlisting{./capitolo_3/solve_t_inf_udiag.m}

che viene utilizzata passando la matrice $A$ e il vettore dei termini noti $\mathbf{b}$ che verrà sovrascritto con il vettore $\mathbf{x}$ via via calcolato dentro la function. 
Un esempio di utilizzo, per risolvere il sistema $A\mathbf{x}=\mathbf{b}$ con
	\[	
		A = 
		\begin{pmatrix}
			1&0&0&0&0 \\
			3&1&0&0&0\\
			-1&2&1&0&0\\
			2&-4&-2&1&0\\
			3&3&1&3&1\\
		\end{pmatrix}
		\quad 
		\mathbf{b}=
		\begin{pmatrix}	4 \\ -5 \\ -38 \\ 77 \\ -40 \end{pmatrix}
	\]
	
 è il seguente:
 
\lstinputlisting{./capitolo_3/solve_t_inf_udiag_test.m}

il risultato è il seguente (compresa verifica moltiplicando per la matrice iniziale):

\begin{lstlisting}[frame=single]

>> solve_t_inf_udiag_test

x =    4   -17     0     1    -4

ans =     0     0     0     0     0

\end{lstlisting}



	\subsection{Esercizio 3.2}

Utilizzare l'algoritmo 3.6 del libro per stabilire se le seguenti matrici sono $sdp$ o no:
	\[	
		A_1 = 
		\begin{pmatrix}
			1&-1&2&2 \\
			-1&5&-14&2\\
			2&-14&42&2\\
			2&2&2&65\\
		\end{pmatrix}
	\]
	\[		
		A_2 =
		\begin{pmatrix}
			1&-1&2&2\\
			-1&6&-17&3\\
			2&-17&48&-16\\
			2&3&-16&4\\
		\end{pmatrix}
	\]
\PP
Usando lo script: \\
\lstinputlisting{./capitolo_1/sdp_test.m}
che si appoggia all'algoritmo 3.6:\\
\lstinputlisting{./capitolo_3/LDL.m}
I risultati sono i seguenti:\\
\begin{lstlisting}[frame=single]
sdp_test

A =

     1    -1     2     2
    -1     5   -14     2
     2   -14    42     2
     2     2     2    65

la matrice e' SDP

A =

     1    -1     2     2
    -1     6   -17     3
     2   -17    48   -16
     2     3   -16     4

la matrice NON e' SDP
\end{lstlisting}



	\subsection{Esercizio 3.3}

Scrivere una function Matlab che, avendo in ingresso un vettore $b$ contenente i termini noti del sistema lineare $Ax = b$ con $A$ sdp e l’output dell’Algoritmo 3.6 del libro (matrice $A$ riscritta nella porzione triangolare inferiore con i fattori $L$ e $D$ della fattorizzazione $LDL^{T}$ di $A$), ne calcoli efficientemente la soluzione.
\PP
\lstinputlisting{./capitolo_3/solve_with_ldl.m}



	\subsection{Esercizio 3.4}

Scrivere una function Matlab che, avendo in ingresso un vettore $b$ contenente i termini noti del sistema lineare $Ax = b$ e l’output dell’Algoritmo 3.7 del libro (matrice $A$ riscritta con la fattorizzazione $LU$ con pivoting parziale e il vettore $p$ delle permutazioni), ne calcoli efficientemente la soluzione.
\PP
\lstinputlisting{./capitolo_3/solve_with_lu_pivoting.m}



	\subsection{Esercizio 3.5}

 Inserire alcuni esempi di utilizzo delle due function implementate per i punti 3 e 4, scegliendo per ciascuno di essi un vettore $\tilde{x}$ e ponendo $b = A\tilde{x}$. Riportare $\tilde{x}$ e la soluzione $x$ da essi prodotta. Costruire anche una tabella in cui, per ogni esempio considerato, si riportano il numero di condizionamento di $A$ in norma 2 (usare $cond$ di Matlab) e le quantità $\frac{\norm{r}}{\norm{b}}$ e $\frac{\norm{x-\tilde{x}}}{\norm{\tilde{x}}}$.
\PP
Utilizzando lo script:

\lstinputlisting{./capitolo_3/exercise_5.m}

si ottengono i seguenti risultati:\\
\begin{tabular}{ | c | r | r | r | r | r | }
\hline
        \textbf{m.size} & \textbf{m.cond} & $\mathbf{e}_{LU} $ & $\mathbf{r}_{LU} $  &  $\mathbf{e}_{LDL}$  &  $\mathbf{r}_{LDL} $ \\
\hline
         50    &     8.81061e+13 		&  2.19014e-11  &  2.16078e-16  &  3.98620e-12  &  1.81091e-16  \\
        100    &    1.26931e+14  	&  4.20607e-11  &  2.86586e-16  &  5.56861e-12  &  3.75498e-16  \\
        150    &    1.55650e+23  	&  1.67742e-07  &  2.59114e-16  &  8.49812e-08  &  2.99143e-16  \\
        200    &    1.79158e+15  	&  7.86844e-11  &  2.19720e-16  &  7.63123e-12  &  2.55269e-16  \\
        250    &    7.91931e+16  	&  1.02967e-10  &  6.81237e-16  &  8.40258e-10  &  2.82641e-16  \\
        300    &    7.07840e+15  	&  1.21648e-10  &  5.02170e-16  &  1.72525e-10  &  3.61705e-16  \\
        350    &    7.83111e+17  	&  2.42229e-09  &  3.73378e-16  &  3.27147e-09  &  4.57571e-16  \\
        400    &    7.25513e+18  	&  3.86121e-10  &  6.12285e-16  &  1.24177e-09  &  3.39125e-16  \\
        450    &    2.18150e+20  	&  8.55091e-09  &  4.06969e-16  &  7.02023e-09  &  5.46399e-16  \\
        500    &    1.37680e+17  	&  1.94177e-10  &  7.33512e-16  &  1.17556e-10  &  3.54496e-16  \\
\hline
\end{tabular}



	\subsection{Esercizio 3.6}

Sia $A = \begin{pmatrix} \epsilon & 1 \\ 1 & 1  \end{pmatrix}$ con $\epsilon = 10^{-13}$. Definire $L$ triangolare inferiore a diagonale unitaria e $U$ triangolare superiore in modo che il prodotto $LU$ sia la fattorizzazione $LU$ di $A$ e, posto $\mathbf{b} = A\mathbf{e}$, con $\mathbf{e} = (1, 1)^T$, confrontare l'accuratezza della soluzione che si ottiene usando il comando $U\setminus(L\setminus \mathbf{b})$ (Gauss senza pivoting) e il comando $A\setminus \mathbf{b}$ (Gauss con pivoting).
\PP
Una fattorizzazione $LU$ di A è: $L = \begin{pmatrix} 1 & 0 \\ \frac{1}{\epsilon} & 1  \end{pmatrix}, U = \begin{pmatrix} \epsilon & 1 \\ 0 & -\epsilon  \end{pmatrix}$, tuttavia eseguendo $U\setminus(L\setminus \mathbf{b})$ si ottiene $\mathbf{x} = (0.999200722162641, 1.000000000000000)^T$ ed un warning: `\textit{Matrix is close to singular or badly scaled. Results may be inaccurate. RCOND =
1.000000e-26}`. Infatti si verifica subito che numero di condizionamento delle matrici L ed U non sono buoni: entrambi sono dell'ordine di $e+25$.
Lo script usato per questo esercizio è:
\lstinputlisting{./capitolo_3/exercise_6.m}



	\subsection{Esercizio 3.7}
Scrivere una function Matlab specifica per la risoluzione di un sistema lineare con matrice dei coefficienti $A \in \mathcal{R}^{n \times n}$ bidiagonale inferiore a diagonale unitaria di Toeplitz, specificabile con uno scalare $\alpha$. Sperimentarne e commentarne le prestazioni (considerare il numero di condizionamento della matrice) nel caso in cui $n = 12$ e $\alpha =100$ ponendo dapprima $b = (1, 101, \dots , 101)^T$ (soluzione esatta $\tilde{x} = (1, \dots , 1)^T$ ) e quindi $b = 0.1 * (1, 101, \dots , 101)^T$ (soluzione esatta $\tilde{x} = (0.1, \dots , 0.1)^T$).
\PP
La funzione per risolvere una matrice bidiagonale inferiore di Toeplitz è la seguente:
\lstinputlisting{./capitolo_3/solve_bidiagonal_toeplitz.m}
Dato che la matrice bidiagonale inferiore di T. è (se non ho capito male) della forma:
\[
\begin{pmatrix}
	1&0&0&0&0\\
	\alpha&1&0&0&0\\
	0&\alpha&1&0&0\\
	0&0&\alpha&1&0\\
	0&0&0&\alpha&1\\
\end{pmatrix}
\]
non è necessario passare alla funzione una matrice intera ma soltanto il vettore $\mathbf{b}$ ed il parametro $\alpha$.
La risoluzione del sistema passa quindi per la seguente semplice formula: 
\fek{x_i = b_i-\alpha x_{i-1} \qquad i=2,3...n } 
Nella funzione Matlab è quindi possibile riscrivere direttamente il vettore in ingresso $\mathbf{b}$.
L'algoritmo ha quindi una occupazione di memoria pari alle $n+1$ posizioni di memoria date dal vettore in ingresso $\mathbf{b}$ e dal parametro $\alpha$; riguardo al numero di operazioni ad ogni passo si esegue una sottrazione e una moltiplicazione quindi queste saranno $2(n-1)$ flops.
La funzione è stata testata verificando la $\mathbf{\tilde{x}}$ risultato dell'algoritmo rispetto alla $\mathbf{x} = A\setminus\mathbf{b}$:
\lstinputlisting{./capitolo_3/solve_bidiagonal_toeplitz_test.m}



	\subsection{Esercizio 3.8}

Scrivere una function che, dato un sistema lineare sovradeterminato $A\mathbf{x}=\mathbf{b}$ con $A \in \mathcal{R}^{m \times n}, m > n, rank(A) = n$ e $\mathbf{b} \in \mathcal{R}^m$, preso come input $\mathbf{b}$ e l'output dell'algoritmo 3.8 del libro (matrice $A$ riscritta con la parte significativa di $R$ e la parte significativa dei vettori di Householder normalizzati con prima componente unitaria), ne calcoli efficientemente la soluzione nel senso dei minimi quadrati.
\PP
La function è la seguente:
\lstinputlisting{./capitolo_3/solve_with_householder.m}
\TODO{verificare porzione core, che cosa fa e se si può riscrivere}

 
 
	\subsection{Esercizio 3.9}
	
Inserire due esempi di utilizzo della function implementata per il punto 8 e confrontare la soluzione ottenuta con quella fornita dal comando $A\setminus \mathbf{b}$.
\PP
La function \lstinline{solve_with_householder} è stata testata con il seguente script:
\lstinputlisting{./capitolo_3/solve_with_householder_test.m}
Nel primo esempio viene usata una matrice $170 \times 15$ ottenendo un risultato che rispetto a $A\setminus \mathbf{b}$ differisce di un ordine $10^{-12}$, nel secondo esempio si usa la matrice dell'esercizio 3.31 del libro di testo, la differenza qui è dell'ordine di $10^{-15}$.


 
	\subsection{Esercizio 3.10}
	
Scrivere una function che realizza il metodo di Newton per un sistema nonlineare (prevedere un numero massimo di iterazioni e utilizzare il criterio di arresto basato sull’incremento in norma eucliedea). Utilizzare la function costruita al punto 4 per la risoluzione del sistema lineare ad ogni iterazione.
\PP
	La function che implementa il metodo iterativo di Newton è la seguente:
\lstinputlisting{./capitolo_3/newton_nonlinear.m}


 
	\subsection{Esercizio 3.11}
	
Verificato che la funzione $f(x_1,x_2) = x_1^2 + x_2^3 - x_1x_2$ ha un punto di minimo relativo in $(1/12, 1/6)$, costruire una tabella in cui si riportano il numero di iterazioni eseguite, e la norma eucliedea dell’ultimo incremento e quella dell’ errore con cui viene approssimato il risultato esatto utilizzando la function sviluppata al punto precedente per valori delle tolleranze pari a $10^{-t}$, con $t=3,6$. Utilizzare $(1/2, 1/2)$ come punto di innesco, verificare che la norma dell'errore è molto più piccola di quella dell'incremento (come mai?).
\PP
Utilizzando lo script:
\lstinputlisting{./capitolo_3/exercise_3_11.m}
si verifica che il punto $(1/12, 1/6)$ è stazionario e, utilizzando una versione modificata della function all'esercizio precedente, si estrae la tabella delle norme di incremente e di errore ad ogni iterazione.
La funzione modificata ha soltanto la firma diversa e restituisce in output gli array delle norme di incremento e di errore:
\begin{lstlisting}[frame=single]
	function [nrs, err] = newton_nonlinear_plot(F, x0, itmax, tol, sol)
	% .....
	while ( k < itmax ) && ( norm(x - x_last) > tol )
        
        k = k + 1;
        nrs(k) = norm(x - x_last);
        err(k) = norm(x - sol);
      	% .....
\end{lstlisting}
Il risultato è: \\
\begin{tabular}{|c|r|r|}
	\hline
	\multicolumn{3}{|c|}{\textbf{t=3}}\\
	\hline
	\textbf{i} & \multicolumn{1}{c|}{$\norm{\mathbf{d_k}}$} & \multicolumn{1}{c|}{$\norm{\mathbf{e_k}}$}\\
	\hline
    		1  &   0.70710678118654  &    0.53359368645273  \\
     		2  &   0.40311288741492  &    0.14907119849998  \\
     		3  &   0.10320313742306  &    0.04586806107691  \\
     		4  &   0.03830219533227  &    0.00756586574464  \\
     		5  &   0.00728174345762  &    0.00028412228702 \\
	\hline
\end{tabular}
\begin{tabular}{|c|r|r|}
	\hline
	\multicolumn{3}{|c|}{\textbf{t=6}}\\
	\hline
	\textbf{i} & \multicolumn{1}{c|}{$\norm{\mathbf{d_k}}$} & \multicolumn{1}{c|}{$\norm{\mathbf{e_k}}$}\\
	\hline
		1  &    0.7071067811865  &    0.5335936864527  \\
     		2  &    0.4031128874149  &    0.1490711984999  \\
    	 	3  &    0.1032031374230  &    0.0458680610769  \\
     		4  &    0.0383021953322  &    0.0075658657446 \\
     		5  &    0.0072817434576  &    0.0002841222870  \\
	     	6  &    0.0002836903857  &    4.3190126192e-07  \\
	\hline
\end{tabular}
	\section{Capitolo 4}

	\subsection{Esercizio 4.1}
Scrivere una function Matlab che implementi il calcolo del polinomio interpolante di grado $n$ in forma di Lagrange. La forma della function deve essere del tipo \texttt{ y = lagrange( xi, fi, x ) }.

\lstinputlisting{./capitolo_4/newton.m}

	\subsection {Esercizio 4.2}
Scrivere una function Matlab che implementi il calcolo del polinomio interpolante di grado $n$ in forma di Newton. La forma della function deve essere del tipo \texttt{ y = newton( xi, fi, x ) }

	\subsection {Esercizio 4.3}
Scrivere una function Matlab che implementi il calcolo del polinomio interpolante di Hermite. La forma della function deve essere del tipo \texttt{ y = newton( xi, fi, f1i, x ) }

	\subsection {Esercizio 4.4}
Utilizzare le functions degli esercizi precedenti per disegnare l'approssimazione della funzione $sin(x)$  nell'intervallo $[0, 2\pi]$, utilizzando le ascisse di interpolazione $x_{i}=i\pi, i=0,1,2$.

	\subsection {Esercizio 4.5}
Scrivere una function Matlab che implementi la \texttt { spline } cubica interpolante (naturale o \textit{not-a-knot}, come specificato in ingresso) delle coppie di dati assegnate. La forma della function deve essere del tipo: \texttt { y = spline3( xi, fi, x, tipo ) }.

	\subsection {Esercizio 4.6}
Scrivere una function Matlab che implementi il calcolo delle ascisse di Chebyshev per il polinomio interpolante di grado $n$, su un generico intervallo $[a,b]$. La function deve essere del tipo: \texttt { xi = ceby( n, a, b ) }.

	\subsection {Esercizio 4.7}
Utilizzare le function degli Esercizi 4.1 e 4.6 per graficare l'approssimazione della funzione di Runge sull'intervallo $[-6,6]$ per $n= 2,4, ... ,40$. Stimare, numericamente, l'errore commesso in funzione del grado $n$ del polinomio interpolante.

	\subsection {Esercizio 4.8}
Relativamente al precedente esercizio, stimare numericamente, la crescita della costante di Lebesgue.

	\subsection {Esercizio 4.9}
Utilizzare la function dell'esercizio 4.1 per approssimare la funzione di Runge sull'intervallo $[-6,6]$, su una partizione uniforme di $n+1$ ascisse, $n= 2,4, ... ,40$. Stimare le corrispondenti costanti di Lebesgue.

	\subsection {Esercizio 4.10}
Stimare, nel senso dei minimi quadrati, posizione, velocità iniziale ed accelerazione relative a un moto rettilineao uniformemente accelerato per cui sono note le seguenti misurazioni delle coppie (tempo, spazio): (1, 2.9), (1, 3.1), (2, 6.9), (2, 7.1), (3, 12.9), (3, 13.1), (4, 20.9), (4, 21.1), (5, 30.9), (5, 31.1).
	\section{Capitolo 5}

	\subsection{Esercizio 5.1}
Scrivere una function Matlab che implementi la formula composita dei trapezi su $n+1$ ascisse equidistanti nell'intervallo $[a, b]$, relativamente alla funzione implementata da \texttt{fun(x)}. 
La function deve essere del tipo: \texttt{ [If] = trapcomp( n, a, b, fun ) }.
La formula composita dei trapezi è così definita:
\begin{equation}\label{trapezi_composite_equation}
	I_1^{(n)} = \frac{b-a}{2n} (f_0 + 2\sum_{i=1}^{n-1}f_i + f_n)
\end{equation}
La function Matlab che implementa la \ref{trapezi_composite_equation} è:
\lstinputlisting{./capitolo_5/trapcomp.m}


\end{document}