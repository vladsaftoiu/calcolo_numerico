\section{Capitolo 4}

	\subsection{Esercizio 4.1}
Scrivere una function Matlab che implementi il calcolo del polinomio interpolante di grado $n$ in forma di Lagrange. La forma della function deve essere del tipo \texttt{ y = lagrange( xi, fi, x ) }.

\lstinputlisting{./capitolo_4/newton.m}

	\subsection {Esercizio 4.2}
Scrivere una function Matlab che implementi il calcolo del polinomio interpolante di grado $n$ in forma di Newton. La forma della function deve essere del tipo \texttt{ y = newton( xi, fi, x ) }

	\subsection {Esercizio 4.3}
Scrivere una function Matlab che implementi il calcolo del polinomio interpolante di Hermite. La forma della function deve essere del tipo \texttt{ y = newton( xi, fi, f1i, x ) }

	\subsection {Esercizio 4.4}
Utilizzare le functions degli esercizi precedenti per disegnare l'approssimazione della funzione $sin(x)$  nell'intervallo $[0, 2\pi]$, utilizzando le ascisse di interpolazione $x_{i}=i\pi, i=0,1,2$.

	\subsection {Esercizio 4.5}
Scrivere una function Matlab che implementi la \texttt { spline } cubica interpolante (naturale o \textit{not-a-knot}, come specificato in ingresso) delle coppie di dati assegnate. La forma della function deve essere del tipo: \texttt { y = spline3( xi, fi, x, tipo ) }.

	\subsection {Esercizio 4.6}
Scrivere una function Matlab che implementi il calcolo delle ascisse di Chebyshev per il polinomio interpolante di grado $n$, su un generico intervallo $[a,b]$. La function deve essere del tipo: \texttt { xi = ceby( n, a, b ) }.

	\subsection {Esercizio 4.7}
Utilizzare le function degli Esercizi 4.1 e 4.6 per graficare l'approssimazione della funzione di Runge sull'intervallo $[-6,6]$ per $n= 2,4, ... ,40$. Stimare, numericamente, l'errore commesso in funzione del grado $n$ del polinomio interpolante.

	\subsection {Esercizio 4.8}
Relativamente al precedente esercizio, stimare numericamente, la crescita della costante di Lebesgue.

	\subsection {Esercizio 4.9}
Utilizzare la function dell'esercizio 4.1 per approssimare la funzione di Runge sull'intervallo $[-6,6]$, su una partizione uniforme di $n+1$ ascisse, $n= 2,4, ... ,40$. Stimare le corrispondenti costanti di Lebesgue.

	\subsection {Esercizio 4.10}
Stimare, nel senso dei minimi quadrati, posizione, velocità iniziale ed accelerazione relative a un moto rettilineao uniformemente accelerato per cui sono note le seguenti misurazioni delle coppie (tempo, spazio): (1, 2.9), (1, 3.1), (2, 6.9), (2, 7.1), (3, 12.9), (3, 13.1), (4, 20.9), (4, 21.1), (5, 30.9), (5, 31.1).