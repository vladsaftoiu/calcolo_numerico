\section{Capitolo 3}


	\subsection{Esercizio 3.1}

Scrivere una function Matlab per la risoluzione di un sistema lineare con matrice dei coefficienti triangolare inferiore a diagonale unitaria. Inserire un esempio di utilizzo.

La function per risolvere il sistema su indicato è la seguente:

\lstinputlisting{./capitolo_3/solve_t_inf_udiag.m}

che viene utilizzata passando la matrice $A$ e il vettore dei termini noti $\mathbf{b}$ che verrà sovrascritto con il vettore $\mathbf{x}$ via via calcolato dentro la function. 
Un esempio di utilizzo, per risolvere il sistema $A\mathbf{x}=\mathbf{b}$ con
	\[	
		A = 
		\begin{pmatrix}
			1&0&0&0&0 \\
			3&1&0&0&0\\
			-1&2&1&0&0\\
			2&-4&-2&1&0\\
			3&3&1&3&1\\
		\end{pmatrix}
	\]
	\[  
		\mathbf{b}=(4,-5,-38,77,-40)
	\]
	
 è il seguente:
 
\lstinputlisting{./capitolo_3/solve_t_inf_udiag_test.m}

il risultato è il seguente (compresa verifica moltiplicando per la matrice iniziale):

\begin{lstlisting}[frame=single]

>> solve_t_inf_udiag_test

x =    4   -17     0     1    -4

ans =     0     0     0     0     0

\end{lstlisting}

	\subsection{Esercizio 3.2}

Utilizzare l'algoritmo 3.6 del libro per stabilire se le seguenti matrici sono $sdp$ o no:
	\[	
		A_1 = 
		\begin{pmatrix}
			1&-1&2&2 \\
			-1&5&-14&2\\
			2&-14&42&2\\
			2&2&2&65\\
		\end{pmatrix}
	\]
	\[		
		A_2 =
		\begin{pmatrix}
			1&-1&2&2\\
			-1&6&-17&3\\
			2&-17&48&-16\\
			2&3&-16&4\\
		\end{pmatrix}
	\]

Usando lo script: \\
\lstinputlisting{./capitolo_1/sdp_test.m}
che si appoggia all'algoritmo 3.6:\\
\lstinputlisting{./capitolo_3/LDL.m}
I risultati sono i seguenti:\\
\begin{lstlisting}[frame=single]
sdp_test

A =

     1    -1     2     2
    -1     5   -14     2
     2   -14    42     2
     2     2     2    65

la matrice e' SDP

A =

     1    -1     2     2
    -1     6   -17     3
     2   -17    48   -16
     2     3   -16     4

la matrice NON e' SDP
\end{lstlisting}


	\subsection{Esercizio 3.3}

Scrivere una function Matlab che, avendo in ingresso un vettore $b$ contenente i termini noti del sistema lineare $Ax = b$ con $A$ sdp e l’output dell’Algoritmo 3.6 del libro (matrice $A$ riscritta nella porzione triangolare inferiore con i fattori $L$ e $D$ della fattorizzazione $LDL^{T}$ di $A$), ne calcoli efficientemente la soluzione.

\lstinputlisting{./capitolo_3/solve_with_ldl.m}



	\subsection{Esercizio 3.4}

Scrivere una function Matlab che, avendo in ingresso un vettore $b$ contenente i termini noti del sistema lineare $Ax = b$ e l’output dell’Algoritmo 3.7 del libro (matrice $A$ riscritta con la fattorizzazione $LU$ con pivoting parziale e il vettore $p$ delle permutazioni), ne calcoli efficientemente la soluzione.

\TODO


	\subsection{Esercizio 3.5}

 Inserire alcuni esempi di utilizzo delle due function implementate per i punti 3 e 4, scegliendo per ciascuno di essi un vettore $\tilde{x}$ e ponendo $b = A\tilde{x}$. Riportare $\tilde{x}$ e la soluzione $x$ da essi prodotta. Costruire anche una tabella in cui, per ogni esempio considerato, si riportano il numero di condizionamento di $A$ in norma 2 (usare $cond$ di Matlab) e le quantità $\frac{\norm{r}}{\norm{b}}$ e $\frac{\norm{x-\tilde{x}}}{\norm{\tilde{x}}}$.

\TODO


	\subsection{Esercizio 3.6}

Sia $A = $ con $\epsilon = 10^{-13}$. Definire $L$ triangolare inferiore a diagonale unitaria e $U$ triangolare superiore in modo che il prodotto $LU$ sia la fattorizzazione $LU$ di $A$ e, posto $b = Ae$, con $e = (1, 1)^T$, confrontare l'accuratezza della soluzione che si ottiene usando il comando $U\setminus(L\setminus b)$ (Gauss senza pivoting) e il comando $A\setminus b$ (Gauss con pivoting).

\TODO


	\subsection{Esercizio 3.7}
Scrivere una function Matlab specifica per la risoluzione di un sistema lineare con matrice dei coefficienti $A \in \mathcal{R}^{n \times n}$ bidiagonale inferiore a diagonale unitaria di Toeplitz, specificabile con uno scalare $\alpha$. Sperimentarne e commentarne le prestazioni (considerare il numero di condizionamento della matrice) nel caso in cui $n = 12$ e $\alpha =100$ ponendo dapprima $b = (1, 101, \dots , 101)^T$ (soluzione esatta $\tilde{x} = (1, \dots , 1)^T$ ) e quindi $b = 0.1 * (1, 101, \dots , 101)^T$ (soluzione esatta $\tilde{x} = (0.1, \dots , 0.1)^T$).

\TODO


	\subsection{Esercizio 3.8}
Scrivere una function che, dato un sistema lineare sovradeterminato $A\mathbf{x}=\mathbf{b}$ con $A \in \mathcal{R}^{m \times n}, m > n, rank(A) = n$ e $\mathbf{b} \in \mathcal{R}^m$, preso come input $\mathbf{b}$ e l'output dell'algoritmo 3.8 del libro (matrice $A$ riscritta con la parte significativa di $R$ e la parte significativa dei vettori di Householder normalizzati con prima componente unitaria), ne calcoli efficientemente la soluzione nel senso dei minimi quadrati.

\TODO


 
	\subsection{Esercizio 3.9}
Inserire due esempi di utilizzo della function implementata per il punto 8 e confrontare la soluzione ottenuta con quella fornita dal comando $A\setminus b$.

\TODO


 
	\subsection{Esercizio 3.10}
Scrivere una function che realizza il metodo di Newton per un sistema nonlineare (prevedere un numero massimo di iterazioni e utilizzare il criterio di arresto basato sull’incremento in norma eucliedea). Utilizzare la function costruita al punto 4 per la risoluzione del sistema lineare ad ogni iterazione.

\TODO


 
	\subsection{Esercizio 3.11}
Verificato che la funzione $f(x_1,x_2) = x_1^2 + x_2^3 - x_1x_2$ ha un punto di minimo relativo in $(1/12, 1/6)$, costruire una tabella in cui si riportano il numero di iterazioni eseguite, e la norma eucliedea dell’ultimo incremento e quella dell’ errore con cui viene approssimato il risultato esatto utilizzando la function sviluppata al punto precedente per valori delle tolleranze pari a $10^{-t}$, con $t=3,6$. Utilizzare $(1/2, 1/2)$ come punto di innesco, verificare che la norma dell'errore è molto più piccola di quella dell'incremento (come mai?).

\TODO